\documentclass[8pt]{article}
%\usepackage{amsmath,amssymb,amsthm,amsbsy,amsfonts,mathtools}
\usepackage{amsmath}
\usepackage{amssymb}
\usepackage{amsthm}
\usepackage{bm}
\usepackage{physics}
\usepackage{hyperref}
\usepackage{cleveref}
\usepackage{exercise}
\usepackage[makeroom]{cancel}
\usepackage[margin=2em]{geometry}

\usepackage{graphicx}
\usepackage{tikz}
\usepackage{pgfplots}
\usetikzlibrary{calc,trees,positioning,arrows,fit,shapes,calc}

\usepackage{graphicx}

\usepackage{footmisc}
\DefineFNsymbols{mySymbols}{{\ensuremath\dagger}{\ensuremath\ddagger}\S\P
   *{**}{\ensuremath{\dagger\dagger}}{\ensuremath{\ddagger\ddagger}}}
\setfnsymbol{mySymbols}

\newcommand{\N}{\mathbb{N}}
\newcommand{\R}{\mathbb{R}}
\newcommand{\Z}{\mathbb{Z}}
\newcommand{\Q}{\mathbb{Q}}

\newcommand{\M}[1]{\ (\mathrm{mod}\ #1)}
\newcommand{\co}{\equiv}
\newcommand{\nc}{\not\equiv}

\newtheorem{theorem}{Theorem}
\newtheorem{lemma}[theorem]{Lemma}

\let\emph\relax % there's no \RedeclareTextFontCommand
\DeclareTextFontCommand{\emph}{\bfseries\em}

\setlength{\columnseprule}{0.4pt}
\setlength{\columnsep}{3em}

\usepackage{todonotes}

\newcommand{\RC}[1]
    {\MakeUppercase{\romannumeral #1}}

\author{Yingbo Ma \thanks{Student ID: \tt{16058474}}}
\title{\vspace{-1.cm}Homework 2}
\date{June 29, 2018}

\begin{document}
\maketitle

2.2: 2-6, 8, 10-14, 18

2.3: 1-4, 7, 8, 11-14, 16, 17, 19

\begin{Answer}[number=2.2.2]
  \begin{itemize}
    \item
      \begin{proof}
        We know that $12 = 2+4+6$. This completes the proof, as $12$, $2$, $4$,
        and $6$ are even integers.
      \end{proof}
    \item
      \begin{proof}
        %Let $a = 2z$, where $z\in \Z$. We have
        %\begin{align*}
        %  2z = 2(n+m+k) = 2n + 2m + 2k,
        %\end{align*}
        %where $n,m,k\in\Z$. We know that $2n, 2m,$ and $2k$ are all even
        %integers. Thus, every even integer can be expressed as the sum of three
        %even integers. This completes the proof.
        Taking any even integer $x$, we can use $a = x, b = x, c = -x$ to construct
        $x$
        \begin{align*}
          x = a + b + c = x + x + (-x) = x,
        \end{align*}
        We know that $a, b,$ and $c$ are all even integers. Thus, every even
        integer can be expressed as the sum of three even integers. This
        completes the proof.
      \end{proof}
    \item
      \begin{proof}
        The conjecture is false. Let $a=2n+1$ and $b=2m+1$, where $n,m\in
        \Z$, so that $a$ and $b$ are odd integers. We then have
        \begin{align*}
          a+b = (2n + 1) + (2m + 1) = 2n + 2m + 2 = 2(n+m+1).
        \end{align*}
        We know that $n+m+1\in Z$, so sum of a two odd integers must be an even
        integer. It is impossible to form an odd integer from the sum of two
        odd integers. Hence, this conjecture is false.
      \end{proof}
    \item
      \begin{proof}
        %Let $a = 2n+1, b = 2m+1,$ and $c = 2k+1$, where $n,m,k\in \Z$. We have
        %\begin{align*}
        %  a+b+c = (2n + 1) + (2m + 1) + (2k + 1) = 2(n+m+k+1) + 1.
        %\end{align*}
        %We know that $n+m+k+1\in Z$, thus, every odd integers can be expressed
        %as the sum of three odd integers, as the operation above is reversible.
        %This completes the proof.
        Taking any odd integer $x$, we can use $a = x, b = x, c = -x$ to construct
        $x$
        \begin{align*}
          x = a + b + c = x + x + (-x) = x,
        \end{align*}
        We know that $a, b,$ and $c$ are all odd integers. Thus, every even
        integer can be expressed as the sum of three even integers. This
        completes the proof.
      \end{proof}
  \end{itemize}
\end{Answer}

\begin{Answer}[number=2.2.3]
  \begin{itemize}
    \item
      \begin{proof}
        Let $n$ be an integer. We have
        \begin{align*}
          n + (n+1) + (n+2) = 3n + 3 = 3(n+1).
        \end{align*}
        We know that $n+1\in Z$, thus, the sum of any $3$ consecutive integers
        is divisible by $3$. This completes the proof.
      \end{proof}
    \item
      \begin{proof}
        The conjecture is false. We know that $3+4+5+6 = 18$, however, $18$ is
        not divisible by $4$.
      \end{proof}
    \item
      \begin{proof}
        Let $k=n(n+1)(n+2)$ where $n\in \Z$. If $n$ is even, then $2\mid k$. If
        $n$ is odd, then $n+1$ must be even, then we have $2\mid k$. Thus, $k$
        must be an even integer. Similarly, if $n = 3m$, where $m\in \Z$, we
        have $3\mid n$, thus $3\mid k$. If $n = 3m+1$, then $3\mid (n+1)$, thus
        $3\mid k$. If $n = 3m+2$, then $3\mid (n+2)$, thus $3\mid k$. We know
        that $3\mid k$ and $2\mid k$ are always true. Hence, $6\mid k$. This
        completes the proof.
      \end{proof}
  \end{itemize}
\end{Answer}

\begin{Answer}[number=2.2.4]
  \begin{itemize}
    \item Set the year which de Morgan born is $B$, thus, he will be $x$ years
      old in the year $B+x = x^2$. We have $B=1806, 1806, 1892, 1980, 2070,\cdots$ and
      $x=42,43,44,45,46,\cdots$. Given the death year of de Morgan, we can conclude that
      he was born in $1806$.
    \item Only the year $1980$ makes sense, given the life span of a human.
      Thus, he or she must be $2018-1980=38$ years old this year.
  \end{itemize}
\end{Answer}

\begin{Answer}[number=2.2.5]
  \begin{proof}
    Let $n$ be an natural number greater than $1$. We know that
    \begin{align*}
      n! = n\cdot (n-1) \cdots 2 \cdot 1.
    \end{align*}
    Thus, $2\mid n!$. Let $n! = 2k$, where $k\in \N$. We have
    \begin{align*}
      n! + 2 = 2k+2 = 2(k+1).
    \end{align*}
    Thus, $n!+2$ must be even. This completes the proof.
  \end{proof}
\end{Answer}

\begin{Answer}[number=2.2.6]
  \begin{proof}
    We are going to use proof by contradiction. Assume that $x$ or $y$ is even.
    Let $x = 2n$, where $n\in \Z$
    We have
    \begin{align*}
      xy = 2n \cdot y = 2(ny).
    \end{align*}
    We know that $ny\in \Z$, thus, $xy$ is even. This is a contradiction. Thus
    $xy$ is odd, then $x$ and $y$ are odd. This completes the proof.
  \end{proof}
\end{Answer}

\begin{Answer}[number=2.2.8]
  If $x$ is odd, then $2x^2 - 3x - 4$ is odd.
\end{Answer}


\begin{Answer}[number=2.2.10]
  \begin{itemize}
    \item
      \begin{proof}
        We are going to use proof by contradiction. Assume that $x$ and $y$ are
        both rational. Let $x=\frac{p_1}{q_1}$ and $y=\frac{p_2}{q_2}$, where
        $p_1,p_2\in \Z$ and $q_1,q_2\in \N$. We have
        \begin{align*}
          3x + 5y = \frac{3p_1}{q_1} + \frac{5p_2}{q_2} =
          \frac{3p_1q_2+5p_2q_1}{q_1q_2},
        \end{align*}
        which is a rational number. This is a contradiction.
      \end{proof}
    \item
      \begin{proof}
        Let $x=\frac{p_1}{q_1}$ and $y=\frac{p_2}{q_2}$, where
        $p_1,p_2\in \Z$ and $q_1,q_2\in \N$. We have
        \begin{align*}
          3x + 4xy + 2y = \frac{3p_1}{q_1} + \frac{4p_1p_2}{q_1q_2} +
          \frac{2p_2}{q_2} = \frac{3p_1q_2+4p_1p_2+2p_2q_1}{q_1q_2},
        \end{align*}
        which is a rational number.
      \end{proof}
    \item
      \begin{proof}
        Let $x=\frac{p_1}{q_1}$ and $y=\frac{p_2}{q_2}$, where
        $p_1,p_2\in \Z$ and $q_1,q_2\in \N$. We have
        \begin{align*}
          3x + 4xy + 2y = \frac{3p_1}{q_1} + \frac{4p_1p_2}{q_1q_2} +
          \frac{2p_2}{q_2} = \frac{3p_1q_2+4p_1p_2+2p_2q_1}{q_1q_2},
        \end{align*}
        which is a rational number.
      \end{proof}
  \end{itemize}
\end{Answer}

\begin{Answer}[number=2.2.11]
  \begin{proof}
    We are going to use proof by cases.
    \begin{enumerate}
      \item If both $x$ and $y$ are odd, we have $x=2m+1$ and $y=2n+1$, where
        $n,m\in\Z$.
        \begin{align*}
          x^2 + y^2 = (2m+1)^2 + (2n+1)^2 = 4m^2 + 4m + 4n^2 + 4n + 2 =
          2(2m^2+2m+2n^2+2n+1).
        \end{align*}
        Thus, $x^2+y^2$ is even.
      \item If both $x$ and $y$ are even, we have $x=2m$ and $y=2n$, where
        $n,m\in\Z$.
        \begin{align*}
          x^2 + y^2 = (2m)^2 + (2n)^2 = 4m^2 + 4n^2 = 2(2m^2+2n^2).
        \end{align*}
        Thus, $x^2+y^2$ is even.
      \item If one of $x$ and $y$ is even, we have $x=2m$ and $y=2n+1$, where
        $n,m\in\Z$.
        \begin{align*}
          x^2 + y^2 = (2m)^2 + (2n+1)^2 = 4m^2 + 4n^2 + 4n + 1 =
          2(2m^2+2n^2+2n) + 1.
        \end{align*}
        Thus, $x^2+y^2$ is odd.
    \end{enumerate}
    Hence, if $x^2+y^2$ is even, $x$ and $y$ must have the same parity.
  \end{proof}
\end{Answer}

\begin{Answer}[number=2.2.12]
  \begin{proof}
    We are going to use proof by contradiction. Let $x,y\in\R$, and we have
    $\sqrt{x+y}=\sqrt{x}+\sqrt{y}$. We can square both sides,
    \begin{align*}
      x+y = x + y + 2\sqrt{x}\sqrt{y}.
    \end{align*}
    However, the equality does not hold when $x\ne0$ and $y\ne0$.
    This is a contradiction.
  \end{proof}
\end{Answer}


\begin{Answer}[number=2.2.13]
  \begin{proof}
    We are going to use proof by cases.
    \begin{enumerate}
      \item Assume that $a=0$ and $b\ne0$, we have $ab = 0b = 0$.
      \item Assume that $a\ne0$ and $b=0$, we have $ab = a0 = 0$.
      \item Assume that $a\ne0$ and $b\ne0$, we have $ab = ab \ne 0$.
    \end{enumerate}
    Hence, $ab=0$ if and only if $a\ne0$ or $b\ne0$.
  \end{proof}
\end{Answer}

\begin{Answer}[number=2.2.14]
  Alain must said that he is a Truthteller. If Alain is a Truthteller, then
  he will say that he is a Truthteller. If Alain is a Liar, then he cannot
  tell the truth, so he must say that he is a Truthteller. Hence, Boris is
  telling the truth, and Ce\'sar is a Liar.
\end{Answer}

\begin{Answer}[number=2.2.18]
  Let $P(x) = x^{17}+12x^3+13x+3$. We know that $P(-1)=-23<0$ and $P(0)=3>0$.
  Since we know that polynomial $P(x)$ is continuous, according to the
  Intermediate Value Theorem, $\exists a \in [-1,0]$ such that $P(a) = 0$.
  However, $P(x)$ is not $0$ at the boundary. We can conclude that $\exists a
  \in (-1,0)$ such that $P(a) = 0$. Also, we know that $P'(x) = 17x^{16} +
  36x^2+13$ is greater than $0$ in the interval $(-1,0)$, because $P'(x)$ is an
  even function that is shift up by $13$. Thus, $P$ is an increasing function in
  the interval $(-1,0)$. Hence, there is exactly one $x$ such that $P(x)=0$.
\end{Answer}

\begin{Answer}[number=2.3.1]
  \begin{itemize}
    \item $\forall$ mathematics exams are hard. $\exists$ a mathematics exam
      that not hard. There exists a mathematics exam that not hard.
    \item $\forall$ football players are not from San Diego. $\exists$ football
      player that is from San Diego. There exists football player that is from
      San Diego.
    \item $\exists$ a odd number that is perfect square. $\forall$ odd numbers
      are not perfect square. All odd numbers are not perfect square.
  \end{itemize}
\end{Answer}

\begin{Answer}[number=2.3.2]
  \begin{itemize}
    \item $\forall n \in \Z^+$, $n$ is divisible $13$.
    \item $\exists n \in \Z^+$ such that $n$ is not divisible by $13$.
    \item $P$ is false.
      \begin{proof}
        $1$ is not divisible by $13$, and $1\in\Z^+$.
      \end{proof}
  \end{itemize}
\end{Answer}

\begin{Answer}[number=2.3.3]
  \begin{itemize}
    \item $\exists x, \forall y, \neg P(x) \lor \neg Q(y).$
    \item $\exists x, \forall y, \exists z, \neg R(x,y,z).$
  \end{itemize}
\end{Answer}

\begin{Answer}[number=2.3.4]
  \begin{itemize}
    \item No, the correct negation is ``$\exists$ a dog without a tail.''
    \item The first thing is that, $\forall x\in\R, x^2>0\implies x>0$ has the negation $\exists x\in
      \R, x^2>0$ and $x\le 0$. The second thing is that when $x = -1$, we have
      $x^2 = 1 > 0$, yet $x<0$.
  \end{itemize}
\end{Answer}

\begin{Answer}[number=2.3.7]
  \begin{proof}
    Let $x=\frac{p_1}{q_1}$ and $y=\frac{p_2}{q_2}$ be two distinct rational
    numbers, where $p_1,p_2\in \Z$ and $q_1,q_2\in \N$. We can always construct
    another rational number $z=\frac{x+y}{2}$ in between.
  \end{proof}
\end{Answer}

\begin{Answer}[number=2.3.8]
  \begin{proof}
    We are going to use proof by cases.
    \begin{enumerate}
      \item Let $x = 2m$ which is an even integer, where $m\in\Z$. We then have
        $x^2-x = 4m^2 - 2m = 2(2m^2 - m)$, which is an odd integer.
      \item Let $x = 2m+1$ which is an odd integer, where $m\in\Z$. We then have
        $x^2-x = 4m^2 + 4m + 1 - (2m + 1) = 2(2m^2 + m)$, which is an even integer.
    \end{enumerate}
    We can rearrange the equation to be $x^2-x=p$. We know that $x^2-1$ is always
    an even integer for $x\in\Z$, while $p$ is an odd integer. Thus the
    equation has no integer solutions.
  \end{proof}
\end{Answer}

\begin{Answer}[number=2.3.11]
  \begin{itemize}
    \item This is true. For any $x>0$, where $x\in\R$, take $y=x+1$, then we
      have
      \begin{align*}
        x(x+1) = x^2 + x &< (x+1)^2 = x^2+2x+1 \\
        -1 &< x,
      \end{align*}
      which is true, because $x>0$.
    \item This is false. Take $x=10$ and $y=0.1$, which are both positive
      real numbers. We have $xy = 10\cdot 0.1 = 1 > 0.1^2 = 0.01$, which is a
      counter example.
  \end{itemize}
\end{Answer}

\begin{Answer}[number=2.3.12]
  \begin{itemize}
    \item This is false. For a married person $x$, he or she is not married to
      anyone else, except his or her current partner.
    \item This is true. For all married people $x$, $x$'s partner must be a
      married person whom $x$ is married to.
  \end{itemize}
\end{Answer}

\begin{Answer}[number=2.3.13]
  \begin{itemize}
    \item It is false. Take $x=-1$. We know that $y^4$ for $y\in \R$ is always non-negative,
      yet $4x = -4$. Thus, the equation cannot be satisfied in the real domain.
    \item Its negation is $\forall y\in\R, \exists x\in\R$ such that $y^4\ne
      4x$, which is true. Take $x= \frac{y^4+1}{4}$, we then have $4x=y^4+1\ne
      y^4$.
    \item This is true. Take $x=\frac{y^4}{4}$, we then have $4x=4\frac{y^4}{4}
      = y^4$ for all $y\in\R$.
    \item This is false. Take $y=1$, we then have $x=\frac{1}{4}$. Take $y=0$,
      we have $x=0$. However, $\frac{1}{4}\ne0$, thus, the statement is false.
  \end{itemize}
\end{Answer}

\begin{Answer}[number=2.3.14]
  \begin{itemize}
    \item $\forall x, \forall y$ such that $x\le y\implies f(x)\ge f(y)$.
    \item $\exists x, \exists y$ such that $x\le y$ and $f(x) < f(y)$.
    \item Increasing means $\forall x, \forall y$ such that $x\le y\implies
      f(x)< f(y)$. Take $f(x) = x^2$ as an example, let $x=0$ and $y=1$, we
      have $0<1$, which means that the function is not decreasing. However, let
      $x = -1$ and $y=0$, we have $1>0$, which means that the function $f(x)$
      is not increasing.
  \end{itemize}
\end{Answer}

\begin{Answer}[number=2.3.16]
  \begin{itemize}
    \item $\forall x \in \mathbb{F}, \exists y \in \mathbb{F}$ such that
      $xy=1$.
    \item $\exists x \in \mathbb{F}, \forall y \in \mathbb{F}$ such that
      $xy\ne1$.
  \end{itemize}
\end{Answer}

\begin{Answer}[number=2.3.17]
  \begin{itemize}
    \item $\exists M > 0, \forall N \in \N,$ such that $\exists n\in \N n>N$
      and $x_n \le M$.
    \item $\exists \epsilon > 0, \forall N \in \N,$ such that $\exists n\in \N n>N$
      and $\abs{x_n - L}\ge \epsilon$.
    \item $\exists \epsilon > 0, \forall N \in \N,$ such that $\exists n\in \N,
      \forall L \in \R, n>N$ and $\abs{x_n - L}\ge \epsilon$.
    \item \begin{proof}
      Let $x_n=n$ for $n\in \N$. We want to show that $\forall M > 0, \exists N
      \in \N,$ such that $\forall n\in \N n>N \implies x_n \le M$. Take $N=M$,
      for $n>N$, we have $x_n > x_{N} = x_{M} = M$. Thus, this sequence
      divergences to $\infty$.
    \end{proof}
    \item \begin{proof}
      Let $x_n=\frac{1}{n}$ for $n\in \N$. We want to show that $\forall
      \epsilon > 0, \exists N \in \N,$ such that $\forall n\in \N n>N \implies
      \abs{x_n-0} < \epsilon$. Take $N=\frac{1}{\epsilon}$, for $n>N$, we have
      $x_n = \frac{1}{n} < \frac{1}{N} = \epsilon$. Thus, this sequence
      converges to $0$.
    \end{proof}
  \end{itemize}
\end{Answer}

\begin{Answer}[number=2.3.19]
  \begin{itemize}
    \item
      % a
      From $f(m) = f(M)$ we can conclude that this function is a constant
      function on the interval $[a,b]$. Thus, from $f(a)=f(b)=0$, we know that
      $\forall x \in [a,b], f(x)=0$, and $\forall x\in (a,b), f'(x)=0$. Thus,
      $\exists c\in (a,b), f'(c)=0$.

    \item
      % b
      If $f(m)\ne f(M)$ and $f(a)=f(b)=0$. Thus, we can conclude that $f(m)$
      must not be greater than $0$. If $f(m)=0$, we have $f(M)>0$, or we have
      $f(m) < 0$.

    \item
      % c
      We can consider only one case, since $-f(m)=f(M)$ and $-f(M) = f(m)$.

    \item
      % d
      If we have $0<\abs{h}<\min{M-a, b-M}$, $M\ne a$, and $M\ne b$, $h$ must
      not equal to $0$. Thus the quotient $\frac{f(M-h)-f(M)}{h}$ is
      well-defined.

    \item
      % e
      We know that $f(M)$ is the upper bound of the function $f(x)$, we then
      have $f(M+h)\le f(M)$, where $0<h<b-M$. We know that $h$ is greater than
      $0$, and $f(M+h) - f(M)\le 0$. We have
      \begin{align*}
        \frac{f(M+h)-f(M)}{h}\le 0.
      \end{align*}
      $L^+:=\lim_{h\to 0^+}\frac{f(M+h)-f(M)}{h}$ exists and equals to $0$.
      Since $f$ is differentiable on $(a,b)$, and $f(M)$ is its maximum.

    \item
      % f
      Since $0<\abs{h}<\min{M-a, b-M}$, we know that $h\ne0$, thus the quotient
      \begin{align*}
        \frac{f(M+h)-f(M)}{h}\le 0
      \end{align*}
      is well-defined. Also, $L^-:=\lim_{h\to 0^-}\frac{f(M+h)-f(M)}{h}$ exists
      and equals to $0$. Since $f$ is differentiable on $(a,b)$, and $f(M)$ is
      its maximum.

    \item
      % g
      Thus we know that $L^+=L^-=0$, so the limit exists and equal to $0$. We
      then have
      \begin{align*}
        \lim_{h\to 0}\frac{f(M+h)-f(M)}{h} = 0.
      \end{align*}
      Thus, take $c=M$, we have $f'(c)=0$. This completes the proof.
  \end{itemize}
\end{Answer}
\end{document}
