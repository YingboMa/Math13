\documentclass[8pt,twocolumn]{article}
%\usepackage{amsmath,amssymb,amsthm,amsbsy,amsfonts,mathtools}
\usepackage{amsmath}
\usepackage{amssymb}
\usepackage{amsthm}
\usepackage{bm}
\usepackage{physics}
\usepackage{hyperref}
\usepackage{cleveref}
\usepackage{exercise}
\usepackage[makeroom]{cancel}
\usepackage[margin=2em]{geometry}

\usepackage{graphicx}
\usepackage{tikz}
\usepackage{pgfplots}
\usetikzlibrary{calc,trees,positioning,arrows,fit,shapes,calc}

\usepackage{graphicx}

\usepackage{footmisc}
\DefineFNsymbols{mySymbols}{{\ensuremath\dagger}{\ensuremath\ddagger}\S\P
   *{**}{\ensuremath{\dagger\dagger}}{\ensuremath{\ddagger\ddagger}}}
\setfnsymbol{mySymbols}

\newcommand{\N}{\mathbb{N}}
\newcommand{\R}{\mathbb{R}}
\newcommand{\Z}{\mathbb{Z}}
\newcommand{\Q}{\mathbb{Q}}

\newcommand{\M}[1]{\ (\mathrm{mod}\ #1)}
\newcommand{\co}{\equiv}
\newcommand{\nc}{\not\equiv}

\newtheorem{theorem}{Theorem}
\newtheorem{lemma}[theorem]{Lemma}

\let\emph\relax % there's no \RedeclareTextFontCommand
\DeclareTextFontCommand{\emph}{\bfseries\em}

\setlength{\columnseprule}{0.4pt}
\setlength{\columnsep}{3em}

\usepackage{todonotes}

\newcommand{\RC}[1]
    {\MakeUppercase{\romannumeral #1}}

\author{Yingbo Ma \thanks{Student ID: \tt{16058474}}}
\title{\vspace{-1.cm}Homework 3}
\date{July 02, 2018}

\begin{document}
\maketitle

3.1: 1, 2, 5-8, 10, 13, 14
3.2: 5-10, 12

\paragraph{3.1.1}
  \begin{align*}
    3^{32} &\co 94143178827 \co 2 \M{7} \\
    3^2 &\co 9 \co 2 \M{7} \\
    3^{32} &\co 3^2 \M{7}
  \end{align*}

\paragraph{3.1.2}
  \begin{align*}
    12^{9} + 19^{24} &\co 2^9 + (-1)^{24} \M{10} \\
    &\co 512 + 1 \co 3 \M{10}
  \end{align*}

\paragraph{3.1.5}
They are identical because, from the \textbf{Theorem 3.7}, we have
  \begin{align*}
    n^2 \nc n &\iff n\co 2\M{3} \\
    \neg (n^2 \nc n ) &\iff \neg (n\co 2\M{3}) \\
    n^2 \co n &\iff ( n\co 0\M{3} \lor n\co 1\M{3} ).
  \end{align*}

\paragraph{3.1.6}
\begin{proof}
  If $a\co b\M{n}$ and $c\co d\M{n}$, we have
  \begin{align*}
    3a \co 3b \M{n} &\land c^2 \co d^2 \M{n} \\
    3a-c^2 &\co 3b - d^2 \M{n}.
  \end{align*}
\end{proof}

\paragraph{3.1.7}
Let $n=1, a=2, b=4$, we have $a^2 \co 4 \M{12} \nc b^2 \co 16 \co 4 \M{12}$,
however, we have $a\co 2 \M{12}$ and $b\co 4\M{12}$.

\paragraph{3.1.8}
\begin{enumerate}
  \item
    \begin{proof}
      Let $n=\sum_{i=1}^k d_i\cdot10^{i-1}$ be an integer with $k$ digits. We then
      have
      \begin{align*}
        n &\co \sum_{i=1}^k d_i\cdot10^{i-1} \M{9} \co \sum_{i=1}^k d_i\cdot1^{i-1}
        \M{9} \\
        &\co \sum_{i=1}^k d_i \M{9}.
      \end{align*}
    \end{proof}
  \item $1+2+3+4+5+6+7+8+9\co45\co0\M{9}$, thus $9\mid123456789$.
\end{enumerate}

\paragraph{3.1.10}
\begin{enumerate}
  \item
    We have $42 \mid 28-7x \implies 6\mid 4-x$, thus $x\co 4\M{6}$.
  \item
    When $x = 4$, we have $7x\co 28\M{42}$ and $x\co 4\M{42}$.
  \item
    No, when $x = 10$, we have $7x\co 28\M{42}$, but $x\co 10\M{42}$.
  \item
    \begin{proof}
      From $ka\co kb \co \M{kn}$, we have $kn\mid kb-ka$, we can divide $k$ both
      sides, we have
      \begin{align*}
        n\mid(b-a) \iff a\co b \co \M{n}.
      \end{align*}
    \end{proof}
\end{enumerate}

\paragraph{3.1.13}
\begin{enumerate}
  \item \begin{proof}
    This is true. Give that $a\mid b$ and $a\mid c$, so that we have $b=q_1 a$
    and $c=q_2 a$, where $q\in \Z$. Thus, we have $bc = q_1a q_2a = (q_1a
    q_2)a$, and $a\mid bc$.
  \end{proof}
  \item \begin{proof}
    This is false. Let $a=3,b=30,c=30$, we have $a\mid c$ and $b\mid c$.
    However, $ab=90$ which does not divides $c$.
  \end{proof}
\end{enumerate}

\paragraph{3.1.14}
\begin{enumerate}
  \item \begin{proof}
      For any integer $b$, if $b\co 0\M{p}$, where $p$ is a prime number, we
      have $b^p\co0\co b\M{p}$. If $b\nc 0\M{p}$, by Fermat's Little Theorem,
      we have $b^{p-1}\co 1\M{p}$. Multiple both sides by $b\M{p}$, we have
      $b^{p}\co b\M{p}$.
  \end{proof}
  \item \begin{proof}
      For any prime number $p$, we have
      $2^{p}\co 2\M{p}\iff 2\co2^{p-1}\M{p}\iff p\mid (2^p-2)$.
  \end{proof}
  \item \begin{proof}
    Let $n=341$, which is not a prime, as $341=11\cdot31$. We have
    $2^{341}-2=4479489484355608421114884561136\\
              8885562432909944692990697999782019275837423603218\\
              90761754986543214231550$, which is dividable by $341$.
  \end{proof}
\end{enumerate}

\paragraph{3.2.5}
If $3x\co 5\M{6}$, we have $6x\co 10\M{6}\co 4\M{6}$ which is impossible, since
$6x\co 0\M{6}$ for $x\in\Z$.

\paragraph{3.2.6}
$12x\co1\M{17}\implies 12x+17y=1$ using Euclidean algorithm, we have
\begin{align*}
  17 &= 1\cdot 12 + 5 \\
  12 &= 2\cdot 5  + 2 \\
  5  &= 2\cdot 2  + 1 \\
  2  &= 2\cdot 1  + 0
\end{align*}
Reversing the algorithm, gives us
\begin{align*}
  1 &= 5 - 2\cdot 2 = 5 - 2\cdot(12 - 2\cdot5) = 5\cdot 5 - 2\cdot 12\\
  &= 5\cdot (17-12) - 2\cdot 12 = 5\cdot 17 - 7\cdot 12.
\end{align*}
Thus, we have the solution of $x\co -7\M{17}\co10\M{17}$.

\paragraph{3.2.7}
$5x+7y=239$

\paragraph{3.2.8}

\paragraph{3.2.9}

\paragraph{3.2.10}

\paragraph{3.2.12}

\end{document}
